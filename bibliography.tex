\renewcommand{\bibname}{Список использованных источников и литературы}
\bibliographystyle{utf8gost71u}
\bibliography{biblio}
\addcontentsline{toc}{chapter}{Список использованных источников и литературы}

% Список использованных источников и литературы должен содержать сведения об источниках, использованных при выполнении работы.  
% При использовании в работе материалов, заимствованных из литературных источников, цитировании различных авторов, необходимо делать соответствующие ссылки, а в конце работы помещать список использованной литературы. При заимствовании текста цитата приводится в кавычках, а после нее в квадратных скобках указывается ссылка на литературный источник по списку использованной литературы. 
% Оформлять ссылку на источник необходимо не только при использовании цитаты, но при включении в ВКР произвольного изложения заимствованных из литературы принципиальных положений, при этом достаточно в квадратных скобках указать ссылку на литературный источник. 
% В список использованных источников и литературы включаются только те названия, на которые есть ссылки в тексте работы. Количество использованных источников свидетельствует о глубине проработанности поставленной проблемы. В списке используется сквозная нумерация всех источников арабскими цифрами, список печатается с абзацного отступа. Порядок построения списка определяется студентом и научным руководителем. Способы расположения материала в списке литературы могут быть следующие: алфавитный, хронологический, по видам изданий, по характеру содержания, по мере появления в тексте.
% Самым распространенным способом формирования списка является алфавитный, при этом сначала указываются нормативно-правовые акты (в порядке убывания юридической силы), затем группируются источники (литература и периодические издания) на русском языке, затем - на иностранном, в конце приводятся электронные ресурсы.
% Список литературы составляют непосредственно по данным печатного издания или выписывают из каталогов и библиографических указателей полностью, без пропусков каких-либо элементов, сокращений заглавий и т.п. 
% Основными элементами описания литературного источника являются: 
%   - ФИО автора (авторов / редактора);
%   - Наименование произведения (название книги);
%   - Наименование издательства;
%   - Год издания;
%   - Количество страниц в издании. 