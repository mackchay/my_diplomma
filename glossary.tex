\chapter*{Определения, обозначения и сокращения}
\addcontentsline{toc}{chapter}{Определения, обозначения и сокращения}

\newcommand{\customglossaryentry}[2]
{
	\textit{#1} --- #2
}

\customglossaryentry{Узел}{это отдельная операция, выполняемая в рамках выполнения SQL-запроса в базе данных, например, сканирование таблицы, индексное сканирование и сортировка данных.}

\customglossaryentry{Кардинальность}{это количество строк, возвращаемых узлом. Она показывает объем данных, который будет обработан на каждом этапе выполнения запроса.}

\customglossaryentry{Селективность}{это показатель того, как эффективно условие фильтрации ограничивает объем данных. Она измеряет долю строк, удовлетворяющих критерию, и помогает оценить, насколько выборочное условие запроса.}

\customglossaryentry{Регрессия}{это ухудшение производительности запросов или неправильная работа функций после обновления версии базы данных или изменений в конфигурации, данных или запросах.}

\customglossaryentry{СУБД}{это  система управления базами данных.}

\customglossaryentry{Бенчмарк}{это тест, применяемый для оценки производительности системы, приложения, оборудования или отдельных компонентов. Он позволяет получить объективные и стандартизированные показатели эффективности, сравнить их с другими решениями или установленными нормами.}

% Определения, обозначения и сокращения – это перечень условных обозначений, символов, принятых в работе сокращений, терминов. Если в работе принята специфическая терминология, а также употребляются мало распространенные сокращения, новые символы, обозначения и т.п., то их перечни должны быть представлены в работе в виде отдельных списков. Перечень должен располагаться столбцом, в котором слева приводят, например, сокращения, справа – его детальную расшифровку. Если в работе специальные термины, сокращения, символы, обозначения и т.п. повторяются не более трех раз, перечень не составляют, а их расшифровку приводят в тексте при первом упоминании. Запись определений, обозначений и сокращений идет в порядке упоминания в тексте работы с необходимой расшифровкой и пояснениями.