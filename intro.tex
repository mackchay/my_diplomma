\chapter*{Введение}
\addcontentsline{toc}{chapter}{Введение}

В современных информационных системах базы данных играют ключевую роль, обеспечивая надежное хранение, управление и быстрый доступ к данным. Одной из наиболее популярных систем управления базами данных является PostgreSQL. С каждым новым релизом в PostgreSQL внедряются улучшения, новые функции и оптимизации, направленные на повышение производительности и безопасности. Однако, несмотря на значительные улучшения, каждая новая версия может приводить к неожиданному ухудшению производительности некоторых запросов. 

\textbf{Объектом исследования} является СУБД PostgreSQL 17 версии и ее механизм планирования и выполнения SQL-запросов.
Предметом исследования является производительность запросов с  заданным планом.

\textbf{Актуальность} заключается в том что, поскольку каждая новая версия PostgreSQL может влиять на производительность, особенно при изменении настроек, необходимо разработать инструмент для выявления случаев ухудшения производительности запросов в новых версиях, и сравнить производительность на разных системах и при разных параметрах postgres. Это позволит разработчикам обратить внимание на участки кода, которые могут приводить к регрессии, и предпринять необходимые меры для оптимизации. Таким образом, тестирующее приложение играет важную роль в обеспечении стабильной производительности системы и повышении качества обслуживания пользователей.

\textbf{Целью данной работы} является разработка приложения, которое будет создавать нагрузку в PostgreSQL и осуществлять измерения времени ее выполнения. Приложение будет генерировать готовые схемы таблиц и запросы по заданным параметрам: тип узла, количество таблиц и размер таблиц. Планируется изучить и добавить узлы, отвечающие за сканирование таблиц и выполнение операций соединения таблиц. Таким образом осуществляется проверка регрессии в PostgreSQL. Для достижения цели были поставлены следующие задачи: 
\begin{enumerate}
    \item Исследовать механизм планирования запросов.
    \item Разработать функционал для генерации SQL-запросов, включающих заданные узлы плана.
    \item Разработать тестовое окружение.
    \item Провести тестирование на разных версиях PostgreSQL и при разных настройках.
\end{enumerate}

\textbf{Научная новизна} работы состоит в том, что будут предложены новые методы моделирования нагрузки с предсказуемостью плана выполнения и генерации запросов с учетом селективности и кардинальности.

\textbf{Практическая ценность} разработки заключается в создании ограниченной произвольной нагрузки с заданным масштабом, что позволит анализировать причины ухудшения производительности запросов в конкретных сценариях. Такой инструмент поможет эффективно выявлять и устранять проблемные места в работе системы. 

Работа включает введение, четыре главы и заключение. В первой главе анализируются существующие решения. Вторая глава посвящена описанию механизма построения запросов в PostgreSQL и разработке архитектуры приложения. В третьей главе будут описаны подробности реализации системы. В четвертой главе представлены результаты тестирования приложения и их анализ. В заключении подводятся основные итоги работы.

