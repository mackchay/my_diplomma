\chapter{Анализ существующих решений проблемы}

В основной части приводят данные, отражающие сущность, методику и основные результаты выполненной работы.

\section{Требования к приложению}

Прежде чем проанализировать существующие приложения для тестирования запросов, были выдвинуты следующие требования:
\begin{enumerate}
	\item \textbf{Создание нагрузки и измерение времени выполнения} — приложение должно быть способно создавать нагрузку на систему и измерять время выполнения запросов, обеспечивая точность анализа производительности.
	\item \textbf{Генерация запросов с широким покрытием случаев} — приложение должно генерировать запросы, охватывающие как можно больше случаев. Однако прямое выполнение всех возможных вариантов может занять слишком много времени и усложнить работу. Для решения этой проблемы можно использовать механизм случайного выбора, регулирующий такие параметры, как селективность запросов, количество столбцов и таблиц и другие характеристики.
	\item \textbf{Классификация запросов по категориям} — приложение должно генерировать запросы, распределенные по категориям. Это позволит анализировать производительность PostgreSQL, выделяя участки, отвечающие за каждую категорию запросов.
\end{enumerate}

\section{Анализ тестирующих приложений для PostgreSQL}

Проанализируем самые популярные приложения для тестирования PostgreSQL.

\subsection{TAP-тесты и Testgres}

\textbf{TAP-тесты (Test Anything Protocol)}. TAP — это протокол для обмена результатами тестов, который активно используется в экосистеме PostgreSQL. TAP-тесты позволяют запускать тесты с использованием стандарта вывода, который поддерживает диагностику и документацию для автоматических систем тестирования. TAP-тесты позволяют интегрировать модульные тесты, автоматически проверять результаты выполнения, а также фиксировать последовательность шагов тестирования. TAP используется для написания тестов, запускаемых с помощью Perl-тестового фреймворка, встроенного в PostgreSQL, или через сторонние инструменты. Используется для интеграционных и модульных тестов PostgreSQL, чтобы обеспечить высокую надежность в разных сборках. TAP полезен при проверке корректности обновлений и изменений в кодовой базе PostgreSQL. TAP часто используется для юнит-тестов и функциональных тестов, где важна проверка ожидаемых результатов, но не тестируются временные характеристики и нагрузочные аспекты работы системы, поэтому TAP-тесты не удовлетворяют требованию создания нагрузки.

\textbf{Testgres.} Testgres — это Python-библиотека для тестирования PostgreSQL, которая предоставляет простой интерфейс для создания и управления экземплярами PostgreSQL, выполнения SQL-запросов и проверки результатов. Testgres позволяет легко создавать, настраивать и удалять временные базы данных PostgreSQL, писать тесты для SQL-запросов и процедур. В дополнение к этому он поддерживает функции, такие как управление репликами и восстановление данных, чтобы тесты могли выполняться в различных конфигурациях базы данных. Подходит для тестов функциональности, поскольку Testgres позволяет симулировать взаимодействие с реальной базой данных, но в контролируемой среде. Библиотека помогает выявлять ошибки в логике SQL-запросов и выявлять потенциальные проблемы до развертывания. Тем не менее Testgres не тестирует производительность запросов, поэтому требованиям он не удовлетворяет.


